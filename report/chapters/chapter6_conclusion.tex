\chapter{Conclusion and Future Work}
\label{ch:conclusion}

\section{Summary of Contributions}

This project has successfully developed a system for extracting causal relationships from World War I historical documents. The key contributions of this work are:

\subsection{Dual-Approach Methodology}

The project implemented two complementary approaches for causal relationship extraction:

\begin{enumerate}
    \item \textbf{Rule-Based Approach:} A linguistically-motivated method using pattern matching, entity extraction, and TF-IDF semantic similarity that extracted 149 high-confidence causal pairs with an average confidence score of 0.93.
    
    \item \textbf{Hybrid ML Approach:} A combined method that integrates rule-based filtering with transformer-based Natural Language Inference (DistilBART-MNLI) that extracted 312 causal pairs with improved coverage while maintaining high precision.
\end{enumerate}

\subsection{Cross-Document Analysis Framework}

The system specifically addresses the challenge of cross-document relationship extraction, identifying causal connections between events described in different source documents. This capability is particularly valuable for historical analysis where events and their consequences are often documented across multiple texts by different authors.

\subsection{WWI-Specific Customization}

The extraction system was customized for the World War I domain with:
\begin{itemize}
    \item Domain-specific entity recognition (battles, locations, military units)
    \item Historical context-aware validation criteria
    \item Appropriate handling of period-specific language patterns
\end{itemize}

\subsection{Visualization and Analysis Tools}

The project provides network visualization capabilities that represent causal relationships as directed graphs, enabling intuitive exploration of the interconnected nature of historical events.

\section{Key Findings}

\subsection{Effectiveness of Hybrid Approach}

The hybrid ML approach demonstrated several advantages over pure rule-based extraction:
\begin{itemize}
    \item More than double the number of extracted relationships (312 vs 149)
    \item Ability to capture relationships with less explicit causal language
    \item Validation through semantic understanding rather than pattern matching alone
\end{itemize}

However, the rule-based approach retained advantages in:
\begin{itemize}
    \item Processing speed (45 seconds vs 8 minutes)
    \item Interpretability of confidence scores
    \item Lower computational resource requirements
\end{itemize}

\subsection{Historical Insights}

The extracted causal networks reveal meaningful patterns in WWI history:
\begin{itemize}
    \item The chain of declarations of war following the assassination of Archduke Franz Ferdinand
    \item The relationship between military operations and diplomatic decisions
    \item Connections between personal experiences (soldier letters) and official historical accounts
    \item The influence of submarine warfare on American entry into the war
\end{itemize}

\subsection{Quality of Extraction}

Manual evaluation of extracted relationships showed:
\begin{itemize}
    \item 90\% of rule-based pairs exhibited clear causal links
    \item 86\% of ML-enhanced pairs exhibited clear causal links
    \item 100\% compliance with cross-document requirements
    \item High relevance to WWI historical context
\end{itemize}

\section{Limitations}

\subsection{Dataset Limitations}

\begin{itemize}
    \item The dataset is limited to English-language documents
    \item Primary focus on British and Australian perspectives
    \item Unbalanced representation of different document types
\end{itemize}

\subsection{Methodological Limitations}

\begin{itemize}
    \item Reliance on explicit and semi-explicit causal indicators
    \item Fixed confidence thresholds may not be optimal for all document types
    \item ML model not specifically fine-tuned for historical causality
\end{itemize}

\subsection{Evaluation Limitations}

\begin{itemize}
    \item No gold-standard annotations for comprehensive evaluation
    \item Limited scale of manual quality assessment
    \item Difficulty in measuring recall without exhaustive annotation
\end{itemize}

\section{Future Work}

\subsection{Short-Term Improvements}

\begin{enumerate}
    \item \textbf{Fine-Tuned Models:} Train a domain-specific NLI model on historical causality examples to improve ML-based validation accuracy.
    
    \item \textbf{Temporal Reasoning:} Incorporate temporal ordering to enforce that causes precede effects chronologically.
    
    \item \textbf{Confidence Calibration:} Develop more sophisticated confidence scoring that better reflects true causal plausibility.
    
    \item \textbf{Interactive Interface:} Create a web-based interface for historians to explore and validate extracted relationships.
\end{enumerate}

\subsection{Medium-Term Extensions}

\begin{enumerate}
    \item \textbf{Multi-Language Support:} Extend the system to handle French, German, and other languages present in WWI documents.
    
    \item \textbf{Causal Chain Inference:} Develop algorithms to infer transitive causal relationships (if A causes B and B causes C, then A indirectly causes C).
    
    \item \textbf{Counterfactual Analysis:} Explore what-if scenarios based on the causal network structure.
    
    \item \textbf{Knowledge Graph Integration:} Connect extracted relationships with external knowledge bases (DBpedia, Wikidata).
\end{enumerate}

\subsection{Long-Term Research Directions}

\begin{enumerate}
    \item \textbf{Causal Strength Quantification:} Develop methods to estimate the strength and importance of different causal links.
    
    \item \textbf{Causal Reasoning:} Use the extracted networks for automated historical reasoning and hypothesis generation.
    
    \item \textbf{Generalization to Other Domains:} Adapt the methodology for other historical periods or domains (medical, legal, scientific texts).
    
    \item \textbf{Explainable Causality:} Develop interpretable models that can explain why extracted relationships are considered causal.
\end{enumerate}

\section{Implications for Digital Humanities}

This project demonstrates the viability of applying modern NLP and ML techniques to historical text analysis. The developed system can serve as a tool for:

\begin{itemize}
    \item \textbf{Historical Research:} Assisting historians in discovering previously unnoticed connections between events
    \item \textbf{Education:} Helping students understand the complex web of causes and effects in historical periods
    \item \textbf{Archive Exploration:} Enabling efficient analysis of large historical document collections
    \item \textbf{Methodology Development:} Providing a foundation for more sophisticated digital humanities tools
\end{itemize}

\section{Concluding Remarks}

The Cross-Document Causal Graph Builder project has successfully demonstrated that combining traditional NLP techniques with modern machine learning can effectively extract meaningful causal relationships from historical texts. The dual-approach methodology provides flexibility, allowing users to choose between the speed and interpretability of rule-based extraction or the improved coverage of the hybrid ML approach.

The extraction of 149 rule-based and 312 ML-enhanced causal pairs from 1,490 WWI documents represents a significant step toward automated historical analysis. The network visualizations created from these relationships offer new ways to explore and understand the interconnected events of World War I.

While limitations remain, particularly in handling implicit causality and in comprehensive evaluation, the project establishes a solid foundation for future research in this area. The combination of linguistic knowledge with deep learning capabilities points toward a promising direction for digital humanities research.

The project fulfills the learning outcomes specified for Student Project 3:
\begin{itemize}
    \item Analysis of the complex task of causal relationship extraction
    \item Proposal of solution strategies combining rule-based and ML approaches
    \item Breakdown of the project into systematic development steps
    \item Application of multiple NLP and ML methods
    \item Scientific presentation of the methodology and results
\end{itemize}

The code, documentation, and extracted results are available for further research and development, contributing to the broader effort of applying computational methods to humanities research.
